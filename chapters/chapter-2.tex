\chapter{Project Management}
\label{chap:project-management}

Common software project management goals that relate to a GDP are as follows \cite{iansommerville2011} -

\begin{itemize}

  \item Timely delivery to the customer
  \item Meeting the customer's expectations
  \item Maintain a well-functioning development team

\end{itemize}

Shakib was the project manager for GDP 8. In this chapter we will analyse how we were able to achieve the above goals and how we could improve our management practices.

\section{Risk Analysis}
\label{sec:risk-analysis}

We will begin with risk analysis, since our management techniques evolved through our efforts of migitating the major risks.

\begin{center}

\begin{longtable}{ |p{5.7cm}|p{2cm}|p{2.2cm}|p{2.7cm}|  }

 \hline
 	%Example of column merging
 	\multicolumn{4}{|c|}{Risk Analysis} \\
 \hline
 	%Examples of centering text
 	\multicolumn{1}{|c|}{Risk} &
 	\multicolumn{1}{|c|}{Affects} &
 	\multicolumn{1}{|c|}{Probability} &
 	\multicolumn{1}{|c|}{Effect}  \\
  \hline
 	 Goals are not clear & Project, Product & Very High & Catastrophic \\
   \hline
   Software not to specification & Product & Very High & Catastrophic \\
   \hline
   Changing requirements & Project, Product & High & Serious \\
  \hline
  Team member unavailable/left & Project & Very High & Serious \\
  \hline
  Technologies unavilable, e.g. VM & Product & Very High & Serious \\
  \hline
  Underperforming software & Product & Low & Tolerable \\
  \hline
  Payment system faulty & Product & Very Low & Catastrophic \\
  \hline
  Continuous integration faulty & Product & Very Low & Tolerable \\
  \hline

\end{longtable}

%Use of captionof allows 'Table' instead of default 'Figure'
\captionof{table}{Risk Analysis}
\label{tab:risk}

\end{center}

In the following sections we discuss how we managed the risks and in turn managed our project.

\section{Time Management}
\label{sec:time-management}
Since a 4th year MEng student should spend 2/3 of their time in GDP, we estimated that, we should each spend at least 30 hours/week (out of 50 hours/week). In our previous experience we found it very difficult to keep to strict hours, because of other commitments, e.g. societies, courseworks etc.\\

To mitigate this risk, we booked a library room \textit{every weekday} for the entire 1st semester. This ensured that we always have a place to work together. We insisted to maximise our contact time, and in fact utilised the room bookings to the fullest by working above \textit{\textbf{30 hours}} together each week (Appendix \ref{appendix:team-member-contribution}).\\

We also estimated the work amount for other coursework in advance, and worked more on GDP in certain weeks. We kept our supervisor and our client well aware of our commitments each week so that we could prepare thusly. Our Gantt Chart in Appendix \ref{appendix:gantt-charts} demonstrates how well we followed our original plan, since work hours closely followed our initial estimates. In fact, the only change is that we finished our initial goals earlier than anticipated and could focus on further achievements like CI.

\section{Weekly Meetings}
\label{sec:weekly-meetings}
We met both the supervisor and the client at least once each week since start of the term. The goals of these meetings were -

\begin{itemize}

  \item Demonstrate the current progress
  \item Prioritise the tasks for the following week
  \item Ensure that the client is satisfied with the direction and progress

\end{itemize}

The meetings mitigated the risk of not meeting the specification and guarded us from veering off to undesirable directions.\\

A key focus during these meetings was to clarify the project goals, deliverables and the success criteria. This made us confident in our progress each week, since we could justify our project direction during the demonstrations based on the agreed success criteria from the previous meetings. We tried to be as empirical as possible in estimating our work during the meetings so that we neither over-estimate, nor under-estimate the difficulty. This, in turn, not only helped us achieve our initial goals, but helped us further extend the deliverables to Synote's first attempt at continuous integration. \\

We also met our client at least once more each week, as well as contact over \textit{Slack} to ensure that we were following his coding conventions, which in turn greatly helped when we merged our code each week.

\section{Team Management}
\label{sec:people-management}
Our project manager, Shakib, focused on the following to guide his team management \cite{iansommerville2011}-

\begin{itemize}

  \item Team members gain new skills
  \item They feel included in the project
  \item Each member is treated idendically
  \item Manager is honest about the team's progress and the individual's contribution

\end{itemize}

The following sections better explain how the team managed the skills and tasks among themselves.

\subsection{Dividing Work}
\label{subsec:dividing-work}
The team started a weekly team meeting after the first progress seminar, after feedback from the supervisor. During this meeting we set our personal weekly tasks. Previously, it was done informally.\\

Usually, after the weekly meeting (Section \ref{sec:weekly-meetings}), we had 4-5 different major tasks for the week. The team members that were present during the team meeting picked the tasks they preferred. Otherwise, the manager assigned a task if any member was absent or did not have a preferrence. Finally, he picked the remaining tasks.\\

During the week, if a team member finished his own task, he started others' remaining tasks if they were unavailable or helped out with ongoing ones. Similarly, if one's task depended on another's task and it was blocking the former's progress, they would either work together, or continue the other's work while they may be unavailable to work. Every effort was made to ensure that no team member spends a development day without a task.

\subsection{Team Skills}
\label{subsec:team-skills}



\subsection{Team Inclusion and Honesty}
\label{subsec:team-inclusion}
The manager, Shakib, emphasised on equality and inclusion of team members in their contribution to the project. Each member was informed of other's contribution through a shared contribution document (Appendix \ref{appendix:team-member-contribution}). He also informed absent team members about group decisions after each meeting so that they feel included. Any discrepancy was openly discussed with the team member so that each team member feel equally valued, and no one is left unfair resentment.\\

Similarly, all team members respected other's responsibilities and even contributed to alleviate other's issues, for example: fill in the hours if someone is unavailable for a job interview.


\section{Code Quality Management}
\label{sec:code-quality-management}


\section{Code Submission Process}
\label{sec:code-submission-process}
