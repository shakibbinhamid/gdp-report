\chapter{Introduction}
\label{chap:introduction}

Our client company Synote has developed a web application to generate automated transcripts for uploaded videos (e.g. lecture videos). The product, also called Synote, originated as a research project at the university and is now being prepared for a much larger audience - general public. We aimed to deliver three mission critical functions to Synote - a payment system, an End to End (E2E) testing framework with E2E tests for the payment system and a Continuous Integration (CI) process implementation. We are proud to have achieved all three goals and client satisfaction (\textbf{Appendix \ref{appendix:client-sign-off}}).\\

Synote lacked a mission critical feature - a payment system for its customers. At the same time, the application had no End to End (E2E) tests. In fact, no Synote developer had succeeded in previous similar efforts. So, our project scope initially included creating a custom E2E testing framework and payment feature for Synote, combined with a demonstration of our framework by writing E2E tests for said payment feature.\\

However, our project scope had since exapanded considerably because of our rapid delivery and meeting of the original goals. In addition to the E2E testing framework, the payment system implementation and E2E tests, we introduced a methodology for code coverage and achieved nearly 100\% server side unit and integration test coverage for the payment system, making it secure and robust.\\

Finally, Synote required a Continuous Integration (CI) process, so that developers could easily and regularly build and test new features. However, Synote had no prior experience in this endeavour. We delivered Synote's first CI process by completely automating the experiment deployment. This has not only automated a manual and error prone task that usually requires 15-20 minutes each time, it has also proven to be greatly helpful in feature development and E2E test reporting.\\

Furthermore, we delivered each of our deliverables on time, even when we had little or no prior experience in technologies used by Synote and most other technologies used in our project itself. We attribute the success to our professional attitude towards project management. We managed our time and skill well from the beginning of the project. Weekly demo and focus meetings with the supervisor and the client assured us of our project direction. Additionally we kept close contact with our client, Synote's lead developer - Yunjia Li, to ensure that all our code is integrated with Synote. We are especially thrilled that our client is already using our testing framework and all of our code is integrated with the Synote's primary repository.\\

We feel that our project was rewarding for both our team and Synote itself, since we believe it has added tangible and mission critical features to Synote. It was a challenging project. So, even with our manifold achievements, we maintain that much work can be done in continuation of our project, which only proves that the project was worth undertaking.\\

Our report will expand on each of the above passages. We begin with our project management in \textbf{Chapter \ref{chap:project-management}} since it will justify how we achieved what follows thereafter. We also introduce necessary background information related to our report in \textbf{Chapter \ref{chap:background-reading}}. Next we cover our deliverables in order of delivery - beginning with the implementation of the payment system in \textbf{Chapter \ref{chap:payment-feature}}, followed by the E2E test framework in \textbf{Chapter \ref{chap:e2e-test-framework}} and CI in \textbf{Chapter \ref{chap:continuous-integration}}. Finally, we discuss how or what we could do differently in terms of future work in \textbf{Chapter \ref{chap:further-work}}.
