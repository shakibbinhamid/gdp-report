\chapter{E2E Test Framework}
\label{chap:e2e-test-framework}

\section{Automated E2E Testing}
\label{sec:automated-e2e-testing}

\subsection{Manual vs Automatic E2E Testing}
\label{sec:manual-vs-automatic-e2e-testing}

\subsection{Technologies Used}
\label{sec:technologies-used}

\section{Test Framework Components}
\label{sec:test-framework-components}

\subsection{Feature Definition as Requirements}
\label{subsec:feautre-definition-as-requirements}

Feature files can be seen as an entry point to cucumber test cases \cite{featurefile1}. Each of the feature file should be written to test a single feature of the application or a particular area of feature \cite{featurefile2}. Since we were writing E2E tests for the payment feature, we created \texttt{buycredits.feature} file for testing this aspect. Consider \textbf{Listing \ref{lst:buycredits-feature-file}}:\\

\begin{listing}[H]
\begin{minted}[xleftmargin=\parindent, linenos, breaklines, breakanywhere, bgcolor=lightgray, fontsize=\small]{cucumber}
@dev @experiment
Feature: Buying Credit feature
  As a user of Synote
  I should be able to successfully buy credits when logged in

  Scenario: Access buy credits page when logged in
    Given I click the "Profile" button on "SideMenu" page
    Given I click the "BuyCredit" button on "Profile" page
    Then I should be on "BuyCredits" page

  Scenario: Buy credit with empty card number
    Given I click the "Profile" button on "SideMenu" page
    Given I click the "BuyCredit" button on "Profile" page
    Given I Correctly type in my details but "" for "number"
    Then Submit button should be disabled
//code omitted
\end{minted}
\captionof{listing}{Section of \texttt{buycredits.feature} file}
\label{lst:buycredits-feature-file}
\end{listing}

The section in \textbf{Listing \ref{lst:buycredits-feature-file}} is written in a language called \texttt{Gherkin}. \texttt{Gherkin} makes use of Business Readable Domain Specific Language (BRDSL) which allows us to write the test cases at a business level without getting into implementation details. 

Our client required us to document the requirements/specification for the new payment feature. Most of Synote's team are developers and since the feature file used \texttt{Gherkin}, it was readable on a business level and could act as automation test script \cite{featurefile1}. Hence, we recommended using the feature files themselves as the requirements documentation.\\

Each of the feature file should be defined with \texttt{Feature} keyword which consists of a name  (see \textbf{Listing \ref{lst:buycredits-feature-file}} line  2) and brief description  (see \textbf{Listing \ref{lst:buycredits-feature-file}} line  3-4). Each of the test cases are defined with \texttt{Scenario} keyword followed by a brief description of that test (see \textbf{Listing \ref{lst:buycredits-feature-file}} line  6 and 11). Each scenario should \cite{featurefile3}:
\begin{itemize}
\item Describe event taking place
\item Describe expected result
\end{itemize}

We use \texttt{Steps} to achieved this. Consider the \texttt{Access buy credits page when logged in} scenario in \textbf{Listing \ref{lst:buycredits-feature-file}}. We use keywords such as \texttt{Given} (line 7 and 8), \texttt{Then} (line 9) for writing readable test cases. Consider \textbf{Table \ref{tab:steps-keywords}} which define \texttt{Step} keywords \cite{featurefile1}

\begin{center}
%Column widths dependent on page width/margins
\begin{tabular}{ |p{2cm}|p{7cm}| }

 \hline
 	Keyword &
 	Description\\
 \hline
 	Given & Describes test pre-condition\\
 \hline
 	And & Defines additional test conditions\\
 \hline
 	Then & Defines expectations of test \\
 \hline

\end{tabular}
\captionof{table}{\texttt{Step keywords}}
\label{tab:steps-keywords}
\end{center}

\subsection{Reusable Steps Definition}
\label{subsec:reusable-steps-definition}

Cucumber is not able to execute the scenarios as they are. Instead, we have to write \texttt{step} definitions. Step definitions use regular expression to map the \texttt{Gherkin steps} to actions which will drive system interactions \cite{stepfile1}. Each of the \texttt{steps} written for scenarios in the \texttt{feature} file should have a \texttt{step} definition declared in its corresponding \texttt{step} file.\\ A \texttt{step} file should only contain definitions for \texttt{steps} used in corresponding \texttt{feature} file.\\

Consider \textbf{Listing \ref{lst:buycredits-step-file}}:\\

\begin{listing}[H]
\begin{minted}[xleftmargin=\parindent, linenos, breaklines, breakanywhere, bgcolor=lightgray, fontsize=\small]{cucumber}
@dev @experiment
//code omitted
	this.Given(/^I click the "([^"]*)" button on "([^"]*)" page$/, function (buttonName, pageName) {
        return this.Support.clickButton(buttonName, pageName);
    });

    this.Given(/^I should be on "([^"]*)" page$/, function (pageName) {
        return this.Support.waitUntil(this.Support.urlChanged(this.Support.getPageUrl(pageName)), 2000)();
    });
//code omitted
\end{minted}
\captionof{listing}{Section of \texttt{buycredits.steps.js} file}
\label{lst:buycredits-step-file}
\end{listing}

In our case, all the \texttt{steps} written in \texttt{buycredits.feature} file are defined in the \texttt{buycredits.steps.js} file. Each step should have a unique definition else an \texttt{ambiguous match} exception will be thrown. \textbf{Listing \ref{lst:buycredits-step-file}} contains the step definitions for \texttt{Access buy credits page when logged in} scenario in \textbf{Listing \ref{lst:buycredits-feature-file}}. Inside the step definitions, we write javascript code to handle the interaction logic e.g. 1st definition in \textbf{Listing \ref{lst:buycredits-step-file}} (line 3) handles clicking of button provided  \texttt{buttonName and pageName} parameters and 2nd definition (line 7) handles checking we are on a page with certain URL provided  \texttt{pageName} parameter. Nearly all of out step definitions take parameters instead of hard-coding them. This way, we can satisfy the \texttt{DRY} principle by reusing generic step definitions, making both  \texttt{feature} and  \texttt{step} files easily maintainable and readable in the long run. 

\subsection{Reusable Support Functions}
\label{subsec:reusable-support-functions}

\subsection{Pre and Post Hooks}
\label{subsec:pre-and-post-hooks}

\subsection{Page Objects}
\label{subsec:page-objects}

\subsection{Deployment Definition}
\label{subsec:deployment-definition}

\subsection{Reporting}
\label{subsec:reporting}

\subsection{Assessment}
\label{subsec:assessment}
