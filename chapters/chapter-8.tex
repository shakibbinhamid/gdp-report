\chapter{Conclusion}
\label{chap:conclusion}

\section{Goals achieved}
\label{sec:goals-achieved}
Over the course of this project, our goals and objectives kept on changing as we kept on meeting our milestones and defined new ones. If at the beginning we sat out to provide our customer with an E2E testing framework and a payment feature, at the end we managed to overachieve and impress our client with a well-rounded continuous integration strategy. A consistent work ethic aided in us in meeting our objectives by setting out clear tasks and operating in an AGILE fashion.

\section{Customer satisfaction}
\label{sec:customer-sastisfaction}
As can be seen in \textbf{Appendix \ref{appendix:client-sign-off}}, the client is satisfied by the work we have done. Apart from achieving what was expected of us and surpassing expectations, we believe a good part of our customer's contentment is due to the fact that we tried to communicate as much as possible with him, giving him updates on our work and requesting feedback and help when needed.

\section{Was the project challenging?}
\label{sec:was-the-project-challenging}
We believe the project was very challenging as we had a limited amount of time to get acquainted with many technologies and practices that were new to us. The workload was not mindless in the slightest and we were required to do extensive research in order to find the path towards a robust solution. Because of this element of unknown, we needed to organise ourselves very well to achieve our goals and put in a lot of hours every week. We faced challenges when mitigating other responsibilities such as coursework, job interviews and personal issues and also technical problems which we needed to overcome.

\section{Was the project fun?}
\label{sec:was-the-project-fun}
We really enjoyed this project from start to finish. We liked that it challenged us from the beginning and that we had to work hard to better ourselves. Learning and experimenting with new technologies kept our interests high and gave us an incentive make as much progress as possible. We appreciated each other's work and company as we pushed forward to get the project done. All in all, we believe it was a wonderful experience.

\section{Location of files}
\label{sec:location-files}
Chapters go in \texttt{chapters} folder. Figures in \texttt{images} folder. No content must be written elsewhere. All content must lie in chapters files'. We may need to create files for sections, if chapters get too big.

\section{Writing Content}
\label{sec:content}
You must not write live on Overleaf! I cannot emphasize it enough! Download the project as a git repo from the options on overleaf. Then write what you need to write, compile locally, then once everything compiles nice and well, put it on overleaf.

\section{Formatting}
\label{sec:formatting}

End each paragraph with a double backslash always.\\

\subsection{Labeling}
\label{subsec:labeling}
Make sure you label 'every component' with a label like above. Chapters will have labels as \texttt{chap:name}, sections as \texttt{sec:name} and so on.

\subsubsection{Naming labels}
\label{subsubsec:naming-labels}
Use all small alphabets in labels, separated by dash (-).\\

This way we can refer any Chapter, Section, Table, Figure, sub components etc.

\subsection{Referring Components}
\label{subsec:referring-components}

You should do bold-face for reference as follows -

\textbf{Section \ref{subsubsec:naming-labels}} is about how to name labels and \textbf{Chapter \ref{chap:introduction}} is the introduction.

\subsection{Figures}

You should place the figures in the \texttt{figures} folder. They should be transparent background in most cases, named all lower cases with dashes. You must label them as \texttt{fig:name}\\

\begin{figure}[!hbt]
  	\centering
 	\includegraphics[width=0.3\textwidth]{exm.jpg}
  	\caption{Normal figure}
 	\label{fig:normal-figure}
\end{figure}

All figures must have captions as well. Sub figures may be left caption less, as long as the entire figure has caption.\\

You can use minipage to insert subfigures as done in \textbf{Figure \ref{fig:subfig-minipage}}. This is the recommended way. Otherwise you can use \textbf{Figure \ref{fig:right-side-minipage}}\\

\begin{figure}[!htb]
    \centering
    \begin{minipage}{.5\textwidth}
        \centering
        \includegraphics[width=0.25\textheight]{exm.jpg}
        \caption{Left side image}
        \label{fig:left side minipage}
    \end{minipage}%
    \begin{minipage}{0.5\textwidth}
        \centering
        \includegraphics[width=0.25\textheight]{exm.jpg}
        \caption{Right side image}
        \label{fig:right-side-minipage}
    \end{minipage}
    \caption{Subfigures using minipage}
    \label{fig:subfig-minipage}
\end{figure}

The subfigure package has caused me problems before, I suggest against it. Similarly, you can use \textbf{Figure \ref{fig:sidecaption-fig}} to have side by side caption or text. You can refer to subfigures as well like \textbf{Figure \ref{fig:left-side-image}} I suggest against it. Use two minipages - one side normal figure, other side text for the figure.

\begin{figure}[!hbt]
    \centering
    \begin{subfigure}[b]{0.4\textwidth}
        \includegraphics[width=\textwidth]{exm.jpg}
        \caption{Left side image}
        \label{fig:left-side-image}
    \end{subfigure}
    ~ %add desired spacing between images, e. g. ~, \quad, \qquad, \hfill etc.
      %(or a blank line to force the subfigure onto a new line)
    \begin{subfigure}[b]{0.4\textwidth}
        \includegraphics[width=\textwidth]{exm.jpg}
        \caption{Right Side Image}
        \label{fig:right-side-image}
    \end{subfigure}
    \caption{Side by side figures using subfigures}
    \label{fig:subfigure}
\end{figure}

\begin{SCfigure}
  \centering
  \caption{Side label with possibly a large amount of text. This is how you would write it}
  \includegraphics[width=0.3\textwidth]{exm.jpg}% filename
  \label{fig:sidecaption-fig}
\end{SCfigure}

\section{Adding Tables}
\label{sec:adding-tables}

Tables must be labeled with usual labeling convention. Also the use of \texttt{multicol} package allows for centering text in table fields and merging of fields as shown in the example below.

Referring to a table example: As show in \textbf{Table \ref{tab:myTable}}. \\

%Column widths dependent on page width/margins
\begin{tabular}{ |p{0.5cm}|p{4cm}|p{4cm}|p{4cm}|  }

 \hline
 	%Example of column merging
 	\multicolumn{4}{|c|}{Table Title Example} \\
 \hline
 	%Examples of centering text
 	\multicolumn{1}{|c|}{id} &
 	\multicolumn{1}{|c|}{Column Heading} &
 	\multicolumn{1}{|c|}{Column Heading} &
 	\multicolumn{1}{|c|}{Column Heading}  \\
 \hline
 	1 & Field 1,1 & Field 1,2 & Field 1,3 \\
 \hline
 	2 & Field 2,1 & Field 2,2 & Field 2,3 \\
 \hline
 	3 & Field 3,1 & Field 3,2 & Field 3,3 \\
 \hline

\end{tabular}
%Use of captionof allows 'Table' instead of default 'Figure'
\captionof{table}{My Table Example}
\label{tab:myTable}
\vspace{0.4cm}

When a larger table is needed that would fit better on a landscape page, the \texttt{rotating} package can be used and a \texttt{sidewaystable} defined as shown in the example on the next page: \textbf{Table \ref{tab:mySidewaysTable}}.

\begin{sidewaystable}
  \begin{tabular}{ |p{0.5cm}|p{7cm}|p{7cm}|p{7cm}|  }
    \hline
        \multicolumn{4}{|c|}{Sideways Table Title Example} \\
     \hline
        \multicolumn{1}{|c|}{id} &
        \multicolumn{1}{|c|}{Column Heading} &
        \multicolumn{1}{|c|}{Column Heading} &
        \multicolumn{1}{|c|}{Column Heading}  \\
     \hline
        1 & Field 1,1 & Field 1,2 & Field 1,3 \\
     \hline
        2 & Field 2,1 & Field 2,2 & Field 2,3 \\
     \hline
        3 & Field 3,1 & Field 3,2 & Field 3,3 \\
     \hline

  \end{tabular}
  \captionof{table}{My Sideways Table Example}
  \label{tab:mySidewaysTable}
\end{sidewaystable}


\section{Code Listing}
\label{sec:code-listing}

You can add code snippets as seen in \textbf{Listing \ref{lst:listing-main}} or \textbf{Listing \ref{lst:side-listing}} full width or mini page listing.

\begin{minipage}{.5\textwidth}
  \begin{listing}[H]
  \begin{minted}[xleftmargin=\parindent, linenos, breaklines, breakanywhere, bgcolor=lightgray, fontsize=\small]{js}

  Name.prototype = {
    methodName: function(params){
      var doubleQuoteString = "some text";
      var singleQuoteString = 'some more text';
      // this is a comment
      if(this.confirmed != null && typeof(this.confirmed) == Boolean && this.confirmed == true){
        document.createElement('h3');
        $('#system').append("This looks great");
        return false;
      } else {
        throw new Error;
      }
    }
  }

  \end{minted}
  \captionof{listing}{caption}
  \label{lst:label-for-listing}
  \end{listing}
\end{minipage}%
\begin{minipage}{0.5\textwidth}
	\centering
	\includegraphics[width=0.25\textheight]{exm.jpg}
	\caption{Right side image}
	\label{fig:right-side-minipage}
\end{minipage}

\begin{listing}[H]
\begin{minted}[xleftmargin=\parindent, linenos, breaklines, breakanywhere, bgcolor=lightgray, fontsize=\small]{js}

Name.prototype = {
  methodName: function(params){
    var doubleQuoteString = "some text";
    var singleQuoteString = 'some more text';
    // this is a comment
    if(this.confirmed != null && typeof(this.confirmed) == Boolean && this.confirmed == true){
      document.createElement('h3');
      $('#system').append("This looks great");
      return false;
    } else {
      throw new Error;
    }
  }
}

\end{minted}
\captionof{listing}{caption}
\label{lst:label-for-listing}
\end{listing}

\section{References}
References are written in \texttt{references.bib} file. I will demo you personally about how to write references. Each reference has a key and you can refer it like this \cite{iansommerville2011}.\\


\section{Directory Trees}
\label{sec:dir-trees}

Directory trees are nice and simple using the package \texttt{dirtree} as shown below in \textbf{Figure \ref{fig:dir-tree}}.

\dirtree{%
  .1 Directory.
  .2 Folder 1.
  .3 Sub Folder 1.
  .4 File 1.
  .2 Folder 2.
}
\captionof{figure}{Directory Tree Example}
\label{fig:dir-tree}
