\chapter{Further Work}
\label{chap:further-work}

% Lack of references due to the assumption that everything here has been touched upon within previous sections

Our interactions with the Synote code base have given us the opportunity to discover its existing strengths and weaknesses, as well as those introduced through our development. Existing weaknesses mostly correspond to the methodology for server side testing whereas the weaknesses we have introduced stem from learning technologies on the fly. With the knowledge we have ascertained at this point, we would propose several enhancements to Synote's design which we shall discuss as future work.\\ 

\subsection{Server Side Testing}
\label{subsec:server-side-testing}

Firstly, we believe that performing integration tests on \texttt{controllers} to test for situations that should be handled by \texttt{policies} is not desirable. The test shown in \textbf{Listing \ref{lst:creditcontroller-test-file}} is intending to check that the user requesting to view credits is himself the credit owner; an action that is performed in a \texttt{policy} named \texttt{canListCreditHistory.js}, not the \texttt{controller}.\\

Our solution is two fold. Firstly, the \texttt{policy} file should be unit tested using mocking and spying techniques. Secondly, the controller actions should be integration tested using dependency injection with a fake database. An example of our suggested \texttt{policy} testing convention is shown in \textbf{Listing \ref{lst:can-list-credit-test}} and demonstrates how the same test (with \texttt{admin} rule extension) can be written using a mocked request stub and a Sinon spy. This approach has the added benefit of running much faster than an \texttt{HTTP} request.  

\begin{listing}[H]
\begin{minted}[xleftmargin=\parindent, linenos, breaklines, breakanywhere, bgcolor=lightgray, fontsize=\small]{js}
it('should not view credit if not credit owner himself', function (done) {
      var user = testdata.getUserByEmail('yl2@ecs.soton.ac.uk');
      var anotheruser = testdata.getUserByEmail('mw@ecs.soton.ac.uk');
      var access_token = testdata.getUserTokenById(anotheruser.id);
      var c;
      Credit.findOne({owner: user.id}).then(function (credit) {
          c = credit;
          return request(sails.hooks.http.app)
              .get('/credit/' + credit.id)
              .set('Authorization', "Bearer " + access_token)
              .expect(403);
      }).then(function (res) {
          done();
      }).catch(done);
});
\end{minted}
\captionof{listing}{\texttt{CreditController.test.js} snippet}
\label{lst:creditcontroller-test-file}
\end{listing}

\begin{listing}[H]
\begin{minted}[xleftmargin=\parindent, linenos, breaklines, breakanywhere, bgcolor=lightgray, fontsize=\small]{js}
    var req, res, cb;

    beforeEach(function (done) {
        req = {
            user_profile: {
                id: '',
                role: ''
            },
            user: {sub: ''}
        };
        res = {};
        cb = sinon.spy();
        done();
    });

    it('If bearer does not match user and not admin, it is forbidden', function (done) {
        req.method = 'GET';
        req.user_profile.id = 'fakeid';
        req.user.sub = 'not matching fakeid';
        req.user.admin = 'not admin';
        res.forbidden = sinon.spy();

        canListCreditHitory(req, res, cb);

        expect(res.forbidden).called;
        expect(cb).not.called;
        done();
    });
\end{minted}
\captionof{listing}{\texttt{canListCreditHistory.test.js} snippet}
\label{lst:can-list-credit-test}
\end{listing}

\subsection{Auto Generated Page Objects}
\label{subsec:auto-generated-page-objects}

We envisioned an extension of our E2E testing framework to auto-generate page object files, but sadly it did not fall into the required feature set. The foundations for this enhancement exist in the form of our HTML ID convention such that a script could be produced to parse an HTML file and generate a page object skeleton based on element IDs. Automatic generation of required test framework skeleton files is also not limited to page objects. Partial templates for feature files and step files could also be generated as part of the envisioned script.\\    

\subsection{E2E Test Reporting}
\label{subsec:e2e-test-reporting}

At present, the E2E testing framework is responsible for generating a folder hierarchy  in which to save a test report based on: the latest Synote commit hash, current deployment and the date-time at which the test suite ran. At an earlier stage the framework was also responsible for posting a link to the generated report on Slack, but we believe the framework should not be responsible for either of these actions and consequently we have moved Slack posting tasks to the Jenkins CI script. The task of separating out the folder hierarchy generation still exists and is recommended as a future task.\\

\subsection{Ghost Inspector}
\label{subsec:ghost-inspector}

Our testing methodology contains a statement: \textit{If a component cannot be tested using our framework, it is considered a bug and should be changed}. In the case of Toastr alerts we adhered to this statement and developed an alternative alerts system. However another component, \texttt{cg-busy}, could also not be tested reliably. This component's use in Synote is desirable and consequently we implemented a test using Ghost Inspector to resolve the testing issue. Including Ghost Inspector in our E2E testing framework would be a fruitful future task resolving the need to compromise on component use.\\ 

\subsection{Testing Controller}
\label{subsec:testing-controller}



\subsection{Server Side Design}
\label{subsec:server-side-design}