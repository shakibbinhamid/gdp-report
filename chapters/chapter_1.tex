\chapter{Introduction}
\label{chap:introduction}

\section{Location of files}
\label{sec:location-files}
Chapters go in \texttt{chapters} folder. Figures in \texttt{images} folder. No content must be written elsewhere. All content must lie in chapters files'. We may need to create files for sections, if chapters get too big.

\section{Writing Content}
\label{sec:content}
You must not write live on Overleaf! I cannot emphasize it enough! Download the project as a git repo from the options on overleaf. Then write what you need to write, compile locally, then once everything compiles nice and well, put it on overleaf.

\section{Formatting}
\label{sec:formatting}

End each paragraph with a double backslash always.\\

\subsection{Labeling}
\label{subsec:labeling}
Make sure you label 'every component' with a label like above. Chapters will have labels as \texttt{chap:name}, sections as \texttt{sec:name} and so on.

\subsubsection{Naming labels}
\label{subsubsec:naming-labels}
Use all small alphabets in labels, separated by dash (-).\\

This way we can refer any Chapter, Section, Table, Figure, sub components etc.

\subsection{Referring Components}
\label{subsec:referring-components}

You should do bold-face for reference as follows -

\textbf{Section \ref{subsubsec:naming-labels}} is about how to name labels and \textbf{Chapter \ref{chap:introduction}} is the introduction.

\subsection{Figures}

You should place the figures in the \texttt{figures} folder. They should be transparent background in most cases, named all lower cases with dashes. You must label them as \texttt{fig:name}\\

\begin{figure}[!hbt]
  	\centering
 	\includegraphics[width=0.3\textwidth]{exm.jpg}
  	\caption{Normal figure}
 	\label{fig:normal-figure}
\end{figure}

All figures must have captions as well. Sub figures may be left caption less, as long as the entire figure has caption.\\

You can use minipage to insert subfigures as done in \textbf{Figure \ref{fig:subfig-minipage}}. This is the recommended way. Otherwise you can use \textbf{Figure \ref{fig:right-side-minipage}}\\

\begin{figure}[!htb]
    \centering
    \begin{minipage}{.5\textwidth}
        \centering
        \includegraphics[width=0.25\textheight]{exm.jpg}
        \caption{Left side image}
        \label{fig:left side minipage}
    \end{minipage}%
    \begin{minipage}{0.5\textwidth}
        \centering
        \includegraphics[width=0.25\textheight]{exm.jpg}
        \caption{Right side image}
        \label{fig:right-side-minipage}
    \end{minipage}
    \caption{Subfigures using minipage}
    \label{fig:subfig-minipage}
\end{figure}

The subfigure package has caused me problems before, I suggest against it. Similarly, you can use \textbf{Figure \ref{fig:sidecaption-fig}} to have side by side caption or text. You can refer to subfigures as well like \textbf{Figure \ref{fig:left-side-image}} I suggest against it. Use two minipages - one side normal figure, other side text for the figure.

\begin{figure}[!hbt]
    \centering
    \begin{subfigure}[b]{0.4\textwidth}
        \includegraphics[width=\textwidth]{exm.jpg}
        \caption{Left side image}
        \label{fig:left-side-image}
    \end{subfigure}
    ~ %add desired spacing between images, e. g. ~, \quad, \qquad, \hfill etc. 
      %(or a blank line to force the subfigure onto a new line)
    \begin{subfigure}[b]{0.4\textwidth}
        \includegraphics[width=\textwidth]{exm.jpg}
        \caption{Right Side Image}
        \label{fig:right-side-image}
    \end{subfigure}
    \caption{Side by side figures using subfigures}
    \label{fig:subfigure}
\end{figure}

\begin{SCfigure}
  \centering
  \caption{Side label with possibly a large amount of text. This is how you would write it}
  \includegraphics[width=0.3\textwidth]{exm.jpg}% filename
  \label{fig:sidecaption-fig}
\end{SCfigure}

\section{Adding Tables}
\label{sec:adding-tables}

Tables must be labeled with usual labeling convention. Also the use of \texttt{multicol} package allows for centering text in table fields and merging of fields as shown in the example below.

Referring to a table example: As show in \textbf{Table \ref{tab:myTable}}. \\

%Column widths dependent on page width/margins
\begin{tabular}{ |p{0.5cm}|p{4cm}|p{4cm}|p{4cm}|  } 

 \hline
 	%Example of column merging
 	\multicolumn{4}{|c|}{Table Title Example} \\
 \hline
 	%Examples of centering text
 	\multicolumn{1}{|c|}{id} & 
 	\multicolumn{1}{|c|}{Column Heading} & 
 	\multicolumn{1}{|c|}{Column Heading} & 
 	\multicolumn{1}{|c|}{Column Heading}  \\
 \hline
 	1 & Field 1,1 & Field 1,2 & Field 1,3 \\
 \hline
 	2 & Field 2,1 & Field 2,2 & Field 2,3 \\
 \hline
 	3 & Field 3,1 & Field 3,2 & Field 3,3 \\
 \hline
 
\end{tabular}
%Use of captionof allows 'Table' instead of default 'Figure' 
\captionof{table}{My Table Example} 
\label{tab:myTable}

\section{Code Listing}
\label{sec:code-listing}

You can add code snippets as seen in \textbf{Listing \ref{lst:listing-main}} or \textbf{Listing \ref{lst:side-listing}} full width or mini page listing.

\begin{minipage}{.5\textwidth}
\begin{lstlisting}[caption=My Javascript Example]
Name.prototype = {
  methodName: function(params){
    var doubleQuoteString = "some text";
    var singleQuoteString = 'some more text';
    // this is a comment
    document.createElement('h3');
    $('#system').append("This looks great");
    return false;
  }
}
\end{lstlisting}
\label{lst:side-listing}
\end{minipage}%
\begin{minipage}{0.5\textwidth}
	\centering
	\includegraphics[width=0.25\textheight]{exm.jpg}
	\caption{Right side image}
	\label{fig:right-side-minipage}
\end{minipage}

\begin{lstlisting}[caption=JS Code Snippet]
Name.prototype = {
  methodName: function(params){
    var doubleQuoteString = "some text";
    var singleQuoteString = 'some more text';
    // this is a comment
    if(this.confirmed != null && typeof(this.confirmed) == Boolean && this.confirmed == true){
      document.createElement('h3');
      $('#system').append("This looks great");
      return false;
    } else {
      throw new Error;
    }
  }
}
\end{lstlisting}
\label{lst:listing-main}

\section{References}
References are written in \texttt{references.bib} file. I will demo you personally about how to write references. Each reference has a key and you can refer it like this \cite{Belk:2014}.\\

