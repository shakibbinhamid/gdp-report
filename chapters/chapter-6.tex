\chapter{Continuous Integration}
\label{chap:continuous-integration}

\section{Synote CI}
\label{sec:synote-ci}

\begin{listing}[H]
\begin{minted}[xleftmargin=\parindent, linenos, breaklines, breakanywhere, bgcolor=lightgray, fontsize=\small]{bash}
wget -q -O - https://pkg.jenkins.io/debian/jenkins-ci.org.key | sudo apt-key add -
sudo sh -c 'echo deb http://pkg.jenkins.io/debian-stable binary/ > /etc/apt/sources.list.d/jenkins.list'
sudo apt-get update
sudo apt-get install jenkins

echo 'upstream jenkins {
  server 127.0.0.1:8080 fail_timeout=0;
}

server {
  listen 80;
  server_name jenkins-cicd.synote.com;
  return 301 https://$host$request_uri;
}

server {
  listen 443 ssl;
  server_name jenkins-cicd.synote.com;

  ssl_certificate /etc/nginx/ssl/synote.com.crt;
  ssl_certificate_key /etc/nginx/ssl/www.synote.com.key;

  location / {
    proxy_set_header        Host $host:$server_port;
    proxy_set_header        X-Real-IP $remote_addr;
    proxy_set_header        X-Forwarded-For $proxy_add_x_forwarded_for;
    proxy_set_header        X-Forwarded-Proto $scheme;
    proxy_redirect http:// https://;
    proxy_pass              http://jenkins;
  }
}' > /etc/nginx/sites-available/jenkins

sudo service nginx restart

\end{minted}
\captionof{listing}{Example from external file}
\label{listing:3}
\end{listing}

\subsection{Setting up Jenkins}
\label{subsec:setting-up-jenkins}

\subsection{Triggering Builds}
\label{subsec:triggering-builds}

\subsection{Synote-Build}
\label{subsec:synote-build}

\subsection{Automated Database Migration}
\label{subsec:automated-database-migration}

\subsection{Restarting Services}
\label{subsec:restarting-services}

\subsection{E2E Testing on Remote}
\label{subsec:e2e-testing-on-remote}

\section{Usefulness of CI}
\label{sec:usefulness-of-ci}
